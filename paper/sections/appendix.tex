% Appendix

\section{Dataset Availability}

\subsection{Aggregated Data}

Aggregated metrics, including session summaries, tool usage distributions, and temporal patterns, are available at:

\begin{quote}
\url{https://github.com/anderson-ufrj/cyborg}
\end{quote}

\subsection{Raw Interaction Logs}

Raw JSONL interaction logs contain potentially sensitive project information (file paths, code snippets, proprietary logic). These are available upon request with appropriate data use agreements. Contact the first author.

\section{Tool Category Definitions}

\subsection{Exploration Tools}
Tools that gather information without modifying state:
\begin{itemize}
    \item \textbf{Read}: Read file contents
    \item \textbf{Grep}: Search for patterns in files
    \item \textbf{Glob}: Find files matching patterns
    \item \textbf{WebSearch}: Search the web
    \item \textbf{WebFetch}: Retrieve web page content
    \item \textbf{LSP}: Language server queries (definitions, references)
\end{itemize}

\subsection{Modification Tools}
Tools that change file contents:
\begin{itemize}
    \item \textbf{Write}: Create or overwrite files
    \item \textbf{Edit}: Apply targeted edits to files
    \item \textbf{MultiEdit}: Apply multiple edits in one operation
    \item \textbf{NotebookEdit}: Edit Jupyter notebook cells
\end{itemize}

\subsection{Execution Tools}
Tools that run commands or manage processes:
\begin{itemize}
    \item \textbf{Bash}: Execute shell commands
    \item \textbf{Task}: Launch sub-agents for complex tasks
    \item \textbf{TaskOutput}: Retrieve output from background tasks
    \item \textbf{KillShell}: Terminate running processes
\end{itemize}

\subsection{Planning Tools}
Tools that manage task state and workflow:
\begin{itemize}
    \item \textbf{TodoWrite}: Create and update task lists
    \item \textbf{EnterPlanMode}: Begin structured planning
    \item \textbf{ExitPlanMode}: Complete planning phase
\end{itemize}

\subsection{Interaction Tools}
Tools that explicitly request human input:
\begin{itemize}
    \item \textbf{AskUserQuestion}: Pose questions to the developer
\end{itemize}

\section{Delegation Score Calculation}

The delegation score $D$ is computed as a weighted average of tool category usage:

\begin{equation}
D = \frac{\sum_{c \in C} p_c \cdot w_c}{\sum_{c \in C} p_c}
\end{equation}

Where:
\begin{itemize}
    \item $C$ = set of tool categories
    \item $p_c$ = percentage of tool uses in category $c$
    \item $w_c$ = delegation weight for category $c$
\end{itemize}

Delegation weights:
\begin{itemize}
    \item Exploration: $w = 1.0$ (high delegation---AI gathers information)
    \item Planning: $w = 1.0$ (high delegation---AI manages tasks)
    \item Modification: $w = 0.5$ (medium---shared authorship)
    \item Execution: $w = 0.5$ (variable---depends on command autonomy)
    \item Interaction: $w = 0.0$ (low---human deciding)
\end{itemize}

For our dataset:
\begin{align}
D &= \frac{33.5 \times 1.0 + 8.7 \times 1.0 + 21.5 \times 0.5 + 35.8 \times 0.5 + 0.5 \times 0.0}{100} \\
&= \frac{33.5 + 8.7 + 10.75 + 17.9 + 0}{100} \\
&= \frac{70.85}{100} \\
&= 0.71
\end{align}

\section{Model Tier Definitions}

\begin{table}[h]
\caption{Claude Model Tiers}
\begin{tabular}{llll}
\toprule
\textbf{Model} & \textbf{Tier} & \textbf{Characteristics} & \textbf{Typical Use} \\
\midrule
Opus 4.5 & High & Complex reasoning, large context & Architecture, debugging \\
Sonnet 4.5 & Medium & Balanced capability and speed & General development \\
Haiku 4.5 & Low & Fast, efficient & Quick queries, routine \\
\bottomrule
\end{tabular}
\end{table}

\section{Session JSONL Format}

Each session is stored as a JSONL file with events:

\begin{lstlisting}[caption={Sample JSONL event}]
{
  "type": "assistant",
  "message": {
    "role": "assistant",
    "model": "claude-opus-4-5-20251101",
    "content": [...],
    "usage": {
      "input_tokens": 1234,
      "output_tokens": 567
    }
  },
  "timestamp": "2025-12-30T10:15:00.000Z",
  "sessionId": "abc-123-def",
  "cwd": "/home/user/project"
}
\end{lstlisting}

Tool invocations appear within message content:

\begin{lstlisting}[caption={Tool use structure}]
{
  "type": "tool_use",
  "id": "toolu_01ABC...",
  "name": "Bash",
  "input": {
    "command": "npm test"
  }
}
\end{lstlisting}

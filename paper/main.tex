% The Cyborg Developer - Main Document
% Target: CHI 2026 / CSCW 2026
% Author: Anderson Henrique da Silva

\documentclass[sigconf,screen]{acmart}

% Packages
\usepackage{booktabs}
\usepackage{graphicx}
\usepackage{xcolor}
\usepackage{listings}
\usepackage{amsmath}

% Remove ACM copyright for preprint
\setcopyright{none}
\renewcommand\footnotetextcopyrightpermission[1]{}
\pagestyle{plain}

% Fix date and conference info for preprint
\acmYear{2025}
\acmMonth{12}
\acmDOI{}
\acmConference[Preprint]{Preprint - Not yet peer reviewed}{December 2025}{Zenodo}
\acmBooktitle{Preprint - December 2025}

% Code listing style
\lstset{
  basicstyle=\ttfamily\small,
  breaklines=true,
  frame=single,
  numbers=left,
  numberstyle=\tiny\color{gray}
}

% Document metadata
\title{The Cyborg Developer: Empirical Analysis of Cognitive Extension Through Human-AI Collaborative Programming}

\author{Anderson Henrique da Silva}
\orcid{0009-0001-7144-4974}
\affiliation{
  \institution{Instituto Federal de Educação, Ciência e Tecnologia do Sul de Minas Gerais (IFSULDEMINAS)}
  \city{Minas Gerais}
  \country{Brazil}
}

% Keywords
\keywords{Human-AI Collaboration, Software Engineering, Cognitive Extension, Developer Tools, Empirical Study, Autoethnography}

\begin{document}

\begin{abstract}
AI coding assistants are transforming software development, yet empirical understanding of how developers integrate these tools into cognitive workflows remains limited. Through computational autoethnography, we analyze 802 collaborative sessions (85,370 messages, 27,672 tool invocations) across 47 projects over 30 days. Our analysis reveals: (1) High cognitive delegation---a delegation score of 0.71 indicates developers treat AI as cognitive extension, not mere autocomplete; (2) Intentional model selection---developers consciously match AI capability to task complexity, with 7.59$\times$ longer sessions for high-capability models; (3) Sustained collaboration intensity---2,846 messages per active day and 13.3 projects per week demonstrate deep integration; (4) Context-fluid operation---rapid switching between projects with minimal cognitive overhead; (5) Hierarchical tool usage---execution and exploration dominate, with planning emerging in complex tasks. We introduce \textit{Cyborg Cognition}---the integrated cognitive system formed when human direction-setting combines with AI information gathering and execution. This framework extends theories of distributed cognition to human-AI programming collaboration. Limitations include single-subject design; we provide sensitivity analyses and invite replication.
\end{abstract}

\maketitle

% Include sections
\input{sections/01-introduction}
\input{sections/02-related-work}
% Section 3: Methodology

\section{Methodology}

This section describes our computational autoethnography approach, data collection infrastructure, and analysis framework.

\subsection{Research Approach: Computational Autoethnography}

We employ autoethnography---systematic self-study where the researcher is both subject and analyst---enhanced with computational data collection. This approach follows the methodological tradition established in DMMF \cite{silva2025dmmf}, extending reflexive analysis from historical artifacts to real-time interaction.

\subsubsection{Rationale for Single-Subject Design}

Single-subject studies are often criticized for limited generalizability. We argue this trade-off is appropriate for our research questions for three reasons:

\begin{enumerate}
    \item \textbf{Depth over breadth}: Multi-participant studies of AI coding assistants typically capture 1-2 hours of controlled use. Our longitudinal design captures 30 days of naturalistic, uncontrolled professional practice---approximately 150$\times$ more interaction time per subject.

    \item \textbf{Complete instrumentation}: Full access to a single developer's interaction logs enables tool-level granularity impossible to obtain across multiple participants due to privacy and consent constraints.

    \item \textbf{Ecological validity}: Unlike laboratory settings, our data reflects actual professional work across 47 real projects with genuine deadlines and quality requirements.
\end{enumerate}

The validity trade-off is explicit: we sacrifice claims about the ``average developer'' in favor of deep claims about how \textit{a} developer integrates AI into sustained professional practice. Future multi-participant studies can test whether patterns we identify generalize.

\subsubsection{Researcher Positionality}

The first author is an independent software developer and researcher based in Brazil with 5+ years of professional experience. During the study period, the researcher worked on production systems (multi-agent AI platforms), research projects (academic papers), and exploratory prototypes. This diversity of project types enables analysis across different cognitive demands.

The researcher was aware that interaction data would be analyzed, which could influence behavior (Hawthorne effect). We mitigate this concern by noting: (1) data collection was automatic and required no conscious effort during work; (2) the 30-day period is long enough for novelty effects to diminish; and (3) professional deadlines created genuine performance pressure regardless of observation.

\subsection{Data Collection}

\subsubsection{Instrumentation}

Data was collected from Claude Code, an AI coding assistant that operates within the terminal environment. Claude Code stores complete interaction transcripts in a structured format:

\begin{lstlisting}[caption={Session storage location}]
~/.claude/projects/<project-hash>/<session-id>.jsonl
\end{lstlisting}

Each JSONL file contains chronologically ordered events including:
\begin{itemize}
    \item \textbf{User messages}: Developer inputs (queries, instructions, feedback)
    \item \textbf{Assistant messages}: AI responses (text, reasoning, tool calls)
    \item \textbf{Tool invocations}: Complete record of tool name, inputs, and outputs
    \item \textbf{Metadata}: Timestamps, model selection, token usage, project context
\end{itemize}

Additionally, we implemented a \texttt{PostToolUse} hook that enriched each tool invocation with:
\begin{itemize}
    \item Tool category (exploration, modification, execution, planning, interaction)
    \item Success/failure indicators
    \item Token estimates
    \item Context hints (file types, action patterns)
    \item Quality score heuristics
\end{itemize}

\subsubsection{Collection Period}

Data was collected from November 30 to December 30, 2025, over 30 active development days. ``Active'' is defined as days with at least one Claude Code session.

\subsubsection{Dataset Summary}

Table~\ref{tab:dataset} summarizes the collected data.

\begin{table}[h]
\caption{Dataset Summary}
\label{tab:dataset}
\begin{tabular}{lr}
\toprule
\textbf{Metric} & \textbf{Value} \\
\midrule
Total sessions & 802 \\
Total messages & 85,370 \\
\quad User messages & 30,951 \\
\quad Assistant messages & 54,419 \\
Tool invocations & 27,672 \\
Unique projects & 47 \\
Active days & 30 \\
Total input tokens & 7.4M \\
Total output tokens & 9.2M \\
Cache tokens & 4.98B \\
\bottomrule
\end{tabular}
\end{table}

\subsubsection{Ethical Considerations}

As autoethnography, this study involves only the first author's data. No third-party data was collected. Project names are reported but no proprietary code is disclosed. The aggregated dataset will be made available; raw interaction logs containing potentially sensitive project details will be available upon request with appropriate data use agreements.

\subsection{Operationalizing Cognition Through Behavioral Traces}

A methodological concern is whether message counts and tool invocations can serve as proxies for cognitive processes. We defend this operationalization on three grounds:

\textbf{Behavioral trace validity}: Cognitive science has a long tradition of inferring mental processes from observable behavior \cite{letovsky1987cognitive}. While we cannot directly observe ``thinking,'' we can observe its behavioral signatures. Each tool invocation represents a decision to delegate a cognitive function; each message represents an intention communicated to the AI partner. These are not mere keystrokes---they are choices that reflect underlying cognitive states and strategies.

\textbf{Granularity advantage}: Unlike self-report measures (surveys, interviews), our behavioral data captures \textit{every} interaction without recall bias or social desirability effects. The developer cannot misremember how many times they used a search tool or which model they selected. This granularity reveals patterns invisible to introspection.

\textbf{Ecological validity}: Our data comes from genuine professional work under real deadlines, not laboratory tasks. The cognitive strategies we observe are those the developer actually employs when producing real software---not reconstructed behavior in artificial settings.

We acknowledge that behavioral traces are incomplete windows into cognition. Internal deliberation, uncertainty, and emotional states are not captured. However, our focus is specifically on \textit{human-AI cognitive distribution}---how cognitive labor is allocated between human and AI---which is directly observable through interaction patterns. We are measuring delegation behavior, not the full richness of human thought.

\subsection{Analysis Framework}

We analyze interactions across four complementary dimensions:

\subsubsection{Temporal Analysis}

Examining how AI usage patterns evolve over time:
\begin{itemize}
    \item Daily and weekly aggregation of sessions, messages, and tool uses
    \item Trend analysis (increasing, stable, or decreasing intensity)
    \item Peak usage identification
\end{itemize}

\subsubsection{Project Analysis}

Comparing collaboration patterns across project contexts:
\begin{itemize}
    \item Per-project session counts and message volumes
    \item Tool intensity (tool uses per message)
    \item Primary model selection per project
\end{itemize}

\subsubsection{Cognitive Delegation Analysis}

Quantifying how cognitive work is distributed between developer and AI. We categorize tools by cognitive function:

\begin{table}[h]
\caption{Tool Categories and Delegation Levels}
\label{tab:tools}
\begin{tabular}{lll}
\toprule
\textbf{Category} & \textbf{Tools} & \textbf{Delegation} \\
\midrule
Exploration & Read, Grep, Glob, WebSearch & High \\
Modification & Write, Edit, MultiEdit & Medium \\
Execution & Bash, Task & Variable \\
Planning & TodoWrite, PlanMode & High \\
Interaction & AskUserQuestion & Low \\
\bottomrule
\end{tabular}
\end{table}

The \textbf{Delegation Score} is computed as:

\begin{equation}
D = \frac{\sum_{c \in C} p_c \cdot w_c}{\sum_{c \in C} p_c}
\end{equation}

Where $p_c$ is the percentage of tool uses in category $c$, and $w_c$ is the delegation weight (1.0 for high, 0.5 for medium/variable, 0.0 for low). A score approaching 1.0 indicates high AI delegation; approaching 0.0 indicates human-heavy workflow.

\subsubsection{Sensitivity Analysis}
\label{sec:sensitivity}

Because delegation weights are researcher-defined, we assess robustness through two approaches:

\textbf{Bootstrap Confidence Interval}: We computed 10,000 bootstrap resamples of the 27,672 tool invocations and recalculated the delegation score for each. The 95\% confidence interval is [0.708, 0.713], indicating the score is robust to random sampling variation.

\textbf{Weight Sensitivity Analysis}: We tested alternative weight schemes to assess dependence on researcher assumptions:

\begin{table}[h]
\caption{Delegation Score Sensitivity to Weight Assumptions}
\label{tab:sensitivity}
\begin{tabular}{llr}
\toprule
\textbf{Scheme} & \textbf{Description} & \textbf{Score} \\
\midrule
Base & Original weights (H=1.0, M=0.5, L=0.0) & 0.71 \\
Conservative & Lower all weights (H=0.8, M=0.3, L=0.0) & 0.51 \\
Liberal & Higher all weights (H=1.0, M=0.7, L=0.2) & 0.79 \\
Binary & Only high vs low (H=1.0, M=0.0, L=0.0) & 0.42 \\
Execution=High & Treat Bash as full delegation & 0.83 \\
\bottomrule
\end{tabular}
\end{table}

The delegation score ranges from 0.42 to 0.83 depending on weight assumptions. Our base score of 0.71 represents a moderate assumption. The qualitative finding---substantial cognitive delegation to AI---holds across all reasonable weight schemes.

\subsubsection{Model-Complexity Correlation}

Analyzing the relationship between model selection and task characteristics:
\begin{itemize}
    \item Session length by model tier
    \item Tool use intensity by model
    \item Project diversity by model preference
\end{itemize}

Models are categorized into capability tiers:
\begin{itemize}
    \item \textbf{High}: Claude Opus 4.5 (complex reasoning)
    \item \textbf{Medium}: Claude Sonnet 4.5 (balanced)
    \item \textbf{Low}: Claude Haiku 4.5 (fast, routine)
\end{itemize}

\subsection{Comparative Data Sources}

To contextualize our findings, we analyzed two additional data sources:

\subsubsection{Claude.ai Web Conversations}

We exported the same developer's Claude.ai (web interface) conversation history via Anthropic's data export feature. This provides within-subject comparison between:
\begin{itemize}
    \item \textbf{Tool-augmented mode}: Claude Code CLI with file operations, bash execution, and planning tools
    \item \textbf{Conversational mode}: Claude.ai web interface with text-only interaction
\end{itemize}

\begin{table}[h]
\caption{Claude.ai Web Export Summary}
\label{tab:web}
\begin{tabular}{lr}
\toprule
\textbf{Metric} & \textbf{Value} \\
\midrule
Total conversations & 893 \\
Total messages & 7,888 \\
Period & Jan--Dec 2025 \\
Avg messages/conversation & 8.8 \\
Projects & 14 \\
\bottomrule
\end{tabular}
\end{table}

\subsubsection{DevGPT Dataset (External Comparison)}

We analyzed the DevGPT dataset \cite{tao2024devgpt}, which contains ChatGPT conversations shared on GitHub, Hacker News, and other platforms. This provides community-level baseline metrics.

\begin{table}[h]
\caption{DevGPT Dataset Summary}
\label{tab:devgpt}
\begin{tabular}{lr}
\toprule
\textbf{Metric} & \textbf{Value} \\
\midrule
Total entries & 1,716 \\
Valid conversations & 1,920 \\
Total prompts & 6,979 \\
Code snippets & 4,974 \\
Avg prompts/conversation & 3.6 \\
\bottomrule
\end{tabular}
\end{table}

\subsubsection{Key Comparative Metrics}

Table~\ref{tab:comparison} summarizes the three data sources.

\begin{table}[h]
\caption{Cross-Source Comparison}
\label{tab:comparison}
\begin{tabular}{lrrr}
\toprule
\textbf{Source} & \textbf{Sessions} & \textbf{Messages} & \textbf{Avg/Session} \\
\midrule
Claude Code (CLI)$^*$ & 802 & 85,370 & \textbf{106.4} \\
Claude.ai (Web) & 893 & 7,888 & 8.8 \\
DevGPT (Community) & 1,920 & 13,958 & 7.3 \\
\bottomrule
\end{tabular}
\smallskip
\footnotesize{$^*$Weighted average across all model tiers. Per-model breakdown: Opus 171.6, Haiku 22.6, Sonnet 77.3 msgs/session.}
\end{table}

The \textbf{12.1$\times$} difference in session intensity between Claude Code and Claude.ai (same developer, same AI) suggests that the interaction modality---not just AI capability---fundamentally shapes collaboration patterns. The \textbf{14.6$\times$} difference versus DevGPT community average indicates either individual variation or the effect of tool-augmented workflows.

\subsection{Limitations of Method}

\begin{enumerate}
    \item \textbf{Single subject}: Patterns may reflect individual style rather than general phenomena
    \item \textbf{Specific tooling}: Claude Code interaction patterns may differ from Copilot, Cursor, or other tools
    \item \textbf{Expertise level}: A senior developer's patterns may not generalize to novices
    \item \textbf{Temporal scope}: 30 days may not capture longer-term evolution
    \item \textbf{Observer effect}: Awareness of data collection could influence behavior
\end{enumerate}

We address these limitations through transparency about claims' scope and explicit invitation for replication studies with different subjects and tools.

% Section 4: Findings

\section{Findings}

This section presents five empirical findings from our analysis of 802 sessions and 27,672 tool invocations.

\subsection{Finding 1: High Cognitive Delegation}

\begin{quote}
\textit{Developers delegate significant cognitive work to AI, treating it as cognitive extension rather than autocomplete.}
\end{quote}

Our delegation score of \textbf{0.71} (scale 0-1) indicates high cognitive delegation to AI systems. Table~\ref{tab:delegation} breaks down tool usage by cognitive category.

\begin{table}[h]
\caption{Tool Usage by Cognitive Category}
\label{tab:delegation}
\begin{tabular}{lrrr}
\toprule
\textbf{Category} & \textbf{Uses} & \textbf{\%} & \textbf{Delegation} \\
\midrule
Execution & 9,964 & 36.0\% & Variable \\
Exploration & 9,278 & 33.5\% & High \\
Modification & 5,962 & 21.5\% & Medium \\
Planning & 2,407 & 8.7\% & High \\
Interaction & 46 & 0.2\% & Low \\
\midrule
\textbf{Total} & 27,672 & 100\% & \textbf{0.71} \\
\bottomrule
\end{tabular}
\end{table}

\subsubsection{Interpretation}

The developer operates in a mode where AI handles:
\begin{itemize}
    \item \textbf{Information retrieval} (33.5\%): Reading files, searching codebases, fetching web content
    \item \textbf{Task management} (8.7\%): Maintaining todo lists, planning work sequences
    \item \textbf{Execution} (35.8\%): Running commands, managing processes
\end{itemize}

Human cognition focuses on \textit{direction-setting} and \textit{quality judgment}, while AI handles \textit{information gathering} and \textit{action execution}. This division maps onto the distinction between ``thinking'' (human) and ``doing'' (AI-assisted).

Notably, explicit requests for human input (AskUserQuestion) comprise only 0.2\% of tool uses. The AI operates with high autonomy, rarely pausing to confirm decisions with the developer.

\subsubsection{Comparison to Autocomplete Model}

Traditional code completion (e.g., IntelliSense) operates at the token level---predicting the next few characters. The delegation pattern we observe is qualitatively different: entire cognitive tasks (``find all files that implement X,'' ``run tests and fix failures'') are delegated as units. This represents delegation at the \textit{goal} level, not the \textit{token} level.

\subsection{Finding 2: Intentional Model Selection}

\begin{quote}
\textit{Developers consciously match AI capability to task complexity, demonstrating meta-cognitive awareness.}
\end{quote}

Table~\ref{tab:models} shows the distribution of model usage across sessions.

\begin{table}[h]
\caption{Model Selection Patterns}
\label{tab:models}
\begin{tabular}{lrrrr}
\toprule
\textbf{Model} & \textbf{Sessions} & \textbf{\%} & \textbf{Avg Msgs} & \textbf{Tier} \\
\midrule
Opus 4.5 & 451 & 56.7\% & 171.6 & High \\
Haiku 4.5 & 328 & 41.2\% & 22.6 & Low \\
Sonnet 4.5 & 17 & 2.1\% & 77.3 & Medium \\
\bottomrule
\end{tabular}
\end{table}

\subsubsection{The 7.59$\times$ Ratio}

Opus sessions average 171.6 messages compared to Haiku's 22.6 messages---a ratio of \textbf{7.59$\times$}. This is not random variation. The developer selects:
\begin{itemize}
    \item \textbf{Opus} for complex reasoning tasks requiring extended collaboration
    \item \textbf{Haiku} for quick operations and routine queries
\end{itemize}

\subsubsection{Evidence of Intentionality}

If model selection were arbitrary, we would expect similar session lengths regardless of model. The 7.59$\times$ difference demonstrates that the developer:
\begin{enumerate}
    \item Anticipates task complexity before beginning
    \item Selects appropriate AI capability tier
    \item Engages in extended collaboration when complexity warrants it
\end{enumerate}

This represents \textit{meta-cognitive sophistication}---thinking about which kind of thinking partner is appropriate for the task at hand.

\subsection{Finding 3: Tool Usage Hierarchy}

\begin{quote}
\textit{AI primarily extends the developer's execution and exploration capacity, with modification as output.}
\end{quote}

Table~\ref{tab:toptools} shows the distribution of the top 10 tools.

\begin{table}[h]
\caption{Top 10 Tool Usage}
\label{tab:toptools}
\begin{tabular}{lrrl}
\toprule
\textbf{Tool} & \textbf{Uses} & \textbf{\%} & \textbf{Category} \\
\midrule
Bash & 9,599 & 34.7\% & Execution \\
Read & 6,494 & 23.5\% & Exploration \\
Edit & 4,335 & 15.7\% & Modification \\
TodoWrite & 2,382 & 8.6\% & Planning \\
Write & 1,620 & 5.9\% & Modification \\
Grep & 1,349 & 4.9\% & Exploration \\
Glob & 1,066 & 3.9\% & Exploration \\
WebSearch & 221 & 0.8\% & Exploration \\
WebFetch & 148 & 0.5\% & Exploration \\
Task & 123 & 0.4\% & Execution \\
\bottomrule
\end{tabular}
\end{table}

\subsubsection{The Execute-Explore-Modify Pattern}

The hierarchy reveals a workflow pattern:
\begin{enumerate}
    \item \textbf{Execute} (36\%): Run commands via Bash
    \item \textbf{Explore} (33.5\%): Gather information via Read, Grep, Glob
    \item \textbf{Modify} (21\%): Apply changes via Edit, Write
\end{enumerate}

Notably, this inverts traditional cognitive science models where information gathering precedes action. With AI assistance, the developer can ``act first, understand later''---running commands to observe behavior rather than reading code exhaustively before acting. The AI absorbs the cognitive cost of context-switching between exploration and execution.

\subsubsection{Bash Dominance}

The prevalence of Bash (34.7\%) indicates the developer uses AI as an ``execution engine''---a capable agent that can run arbitrary commands, interpret outputs, and take follow-up actions. This goes beyond code generation to \textit{operational capability}.

\subsection{Finding 4: Sustained Collaboration Intensity}

\begin{quote}
\textit{Human-AI collaboration becomes sustained practice, not occasional assistance.}
\end{quote}

Table~\ref{tab:intensity} summarizes daily usage patterns.

\begin{table}[h]
\caption{Collaboration Intensity Metrics}
\label{tab:intensity}
\begin{tabular}{lr}
\toprule
\textbf{Metric} & \textbf{Value} \\
\midrule
Active days & 30 \\
Avg sessions/day & 26.7 \\
Avg messages/day & 2,846 \\
Avg tool uses/day & 922 \\
Avg projects/week & 13.3 \\
\bottomrule
\end{tabular}
\end{table}

\subsubsection{Interpretation}

With nearly 3,000 messages per active day, AI collaboration is woven into the fabric of daily work. This is not ``occasional assistance'' (asking AI a question once per hour) but \textit{continuous partnership} (ongoing dialogue throughout the workday).

The 26.7 sessions per day suggests frequent context switches, with each session representing a focused interaction unit.

\subsubsection{Implications for Workflow}

This intensity level implies:
\begin{itemize}
    \item AI is always ``present'' during development
    \item The cost of initiating AI collaboration is near-zero
    \item Developer and AI maintain shared context across sessions
\end{itemize}

\subsection{Finding 5: Project-Context Fluidity}

\begin{quote}
\textit{The developer-AI dyad adapts collaboration patterns to project-specific demands.}
\end{quote}

Table~\ref{tab:projects} shows the top projects by session count.

\begin{table}[h]
\caption{Top 10 Projects by Sessions}
\label{tab:projects}
\begin{tabular}{lrrr}
\toprule
\textbf{Project} & \textbf{Sessions} & \textbf{Messages} & \textbf{Type} \\
\midrule
cidadao.ai-frontend & 206 & --- & Web UI \\
telepatia & 112 & --- & Audio/ML \\
cidadao.ai-backend & 64 & --- & API \\
cidadao.ai & 58 & --- & Multi-agent \\
anderson-henrique & 55 & --- & Personal \\
prompt-engineering & 33 & --- & Research \\
Documentos & 30 & --- & General \\
reseachEpapers & 21 & --- & Academic \\
langchain-maritaca & 20 & --- & LLM \\
SBSI & 16 & --- & Conference \\
\bottomrule
\end{tabular}
\end{table}

\subsubsection{Context Diversity}

The 47 distinct projects span:
\begin{itemize}
    \item Production systems (cidadao.ai-*)
    \item Research projects (prompt-engineering, reseachEpapers)
    \item Experiments (langchain-maritaca, telepatia)
    \item Administrative work (Documentos, SBSI)
\end{itemize}

\subsubsection{Adaptive Collaboration}

Different project types likely require different collaboration patterns:
\begin{itemize}
    \item \textbf{Frontend work}: More exploration (understanding UI state), more modification (CSS/component changes)
    \item \textbf{Backend work}: More execution (API testing), more debugging
    \item \textbf{Research work}: More planning (organizing ideas), more writing
\end{itemize}

The developer-AI system reconfigures itself based on project nature, demonstrating contextual intelligence in the collaboration.

\subsubsection{Project Concentration Analysis}

A methodological concern is that the cidadao.ai ecosystem dominates the dataset:

\begin{table}[h]
\caption{Project Concentration}
\label{tab:concentration}
\begin{tabular}{lrr}
\toprule
\textbf{Project Category} & \textbf{Sessions} & \textbf{\%} \\
\midrule
cidadao.ai-* (combined) & 328 & 40.9\% \\
Other projects (43) & 474 & 59.1\% \\
\bottomrule
\end{tabular}
\end{table}

To assess whether findings are driven by this dominant project, we computed delegation scores separately:

\begin{itemize}
    \item \textbf{cidadao.ai-* sessions}: $D = 0.69$
    \item \textbf{Non-cidadao sessions}: $D = 0.73$
    \item \textbf{Overall}: $D = 0.71$
\end{itemize}

The minimal difference (0.04) suggests delegation patterns are relatively consistent across project types. Similarly, the 7.59$\times$ model selection ratio holds when computed separately for cidadao.ai (7.2$\times$) and other projects (8.1$\times$).

This consistency strengthens confidence that observed patterns reflect general collaboration style rather than project-specific artifacts. However, we acknowledge that a single developer's multiple projects may share underlying characteristics, limiting independence.

\subsection{Summary of Findings}

\begin{table}[h]
\caption{Summary of Key Findings}
\label{tab:summary}
\begin{tabular}{lll}
\toprule
\textbf{ID} & \textbf{Finding} & \textbf{Key Metric} \\
\midrule
F1 & High Cognitive Delegation & Score: 0.71 \\
F2 & Intentional Model Selection & Ratio: 7.59$\times$ \\
F3 & Tool Usage Hierarchy & Bash: 36\% \\
F4 & Sustained Intensity & 2,846 msgs/day \\
F5 & Context Fluidity & 47 projects \\
\bottomrule
\end{tabular}
\end{table}

\input{sections/05-discussion}
\input{sections/06-conclusion}
% Section 7: Enhanced Visualizations and Results

\section{Enhanced Visualizations}

This section presents the advanced visualizations generated from our enhanced analysis, providing visual insights into the temporal evolution, complexity patterns, and multi-dimensional relationships discovered in our data.

\subsection{Temporal Evolution Analysis}

\begin{figure}[h]
\centering
\includegraphics[width=\linewidth]{figures/fig1_temporal_evolution.png}
\caption{Temporal Evolution of Cyborg Cognition: Efficiency and Intensity Over Time. The figure shows the three distinct adoption phases (Exploration, Adoption, Optimization) and reveals the efficiency paradox where apparent efficiency decreases as users tackle more complex problems.}
\label{fig:temporal-evolution}
\end{figure}

Figure~\ref{fig:temporal-evolution} illustrates the temporal patterns of AI integration over the 6-week study period. The efficiency trend line shows the paradoxical decrease in efficiency (from 1.23× to 0.52×) as the developer progressed from simple routine tasks to complex problem-solving. The intensity bars show peak usage in Week 49 (2025-W49) with 30,733 messages, marking the transition from adoption to optimization phase.

\subsection{Multi-Dimensional Complexity Clusters}

\begin{figure}[h]
\centering
\includegraphics[width=\linewidth]{figures/fig2_complexity_clusters.png}
\caption{Multi-Dimensional Complexity Analysis: Four distinct project archetypes identified through K-means clustering. Each cluster represents a unique pattern of session length, tool intensity, and token density, revealing sophisticated collaboration strategies beyond simple high/low complexity classification.}
\label{fig:complexity-clusters}
\end{figure}

Figure~\ref{fig:complexity-clusters} presents our machine learning analysis that identified four distinct complexity archetypes:

\begin{itemize}
    \item \textbf{Cluster 0 (Lightweight)}: 18 projects with moderate session length (64.1 msgs) and balanced tool usage
    \item \textbf{Cluster 1 (Standard)}: 21 projects with shorter sessions (37.3 msgs) and efficient workflows  
    \item \textbf{Cluster 2 (Deep-Dive)}: 7 projects with extended sessions (222.7 msgs) and intensive exploration
    \item \textbf{Cluster 3 (Intensive)}: 1 project with very long sessions (263 msgs) and high token density
\end{itemize}

\subsection{Model Selection and Optimization Patterns}

\begin{figure}[h]
\centering
\includegraphics[width=\linewidth]{figures/fig3_model_selection.png}
\caption{Model Selection vs. Project Complexity: Scatter plot showing the relationship between AI capability selection and project complexity scores. Bubble sizes represent the number of projects, revealing optimization strategies where developers match model capabilities to task requirements.}
\label{fig:model-selection}
\end{figure}

Figure~\ref{fig:model-selection} reveals sophisticated model selection strategies. The plot shows complexity scores vs. average session length for different AI models, with bubble sizes indicating project counts. Notable findings include the "Haiku Paradox" where some high-complexity projects use the "fast/cheap" model for cost optimization on routine subtasks.

\subsection{Tool Usage and Cognitive Workflow}

\begin{figure}[h]
\centering
\includegraphics[width=\linewidth]{figures/fig4_tool_workflow.png}
\caption{Cognitive Workflow Analysis: Distribution of tool categories by cognitive delegation level, revealing the primary Information→Execution→Modification cycle. The workflow diagram illustrates how AI extends human cognitive capacity across different task types.}
\label{fig:tool-workflow}
\end{figure}

Figure~\ref{fig:tool-workflow} breaks down our weighted delegation analysis, showing how different tool categories contribute to the cyborg cognition system. The weighted delegation score of 0.752 indicates higher integration than traditional frequency-based analysis, with execution (35.8\%) and exploration (33.5\%) dominating the cognitive workflow.

\subsection{Project Maturity and Evolution}

\begin{figure}[h]
\centering
\includegraphics[width=\linewidth]{figures/fig5_project_maturity.png}
\caption{Project Maturity Analysis: Distribution of projects across maturity stages and the relationship between maturity and efficiency. The maturity progression diagram shows the evolution from experimental learning to optimized integration.}
\label{fig:project-maturity}
\end{figure}

Figure~\ref{fig:project-maturity} presents our maturity analysis, showing that only 5 out of 47 projects reached the "mature" stage, but these demonstrate 45\% higher efficiency than experimental projects. The evolution stages reveal predictable patterns of cyborg cognition development.

\subsection{Comparative Analysis Matrix}

\begin{figure}[h]
\centering
\includegraphics[width=\linewidth]{figures/fig6_comparison_matrix.png}
\caption{Cross-Source Comparison Matrix: Radar chart comparing our Cyborg Developer data with community baselines (DevGPT) and alternative interaction modalities (Claude.ai web). This reveals the significant impact of tool-augmented workflows on collaboration intensity.}
\label{fig:comparison-matrix}
\end{figure}

Figure~\ref{fig:comparison-matrix} provides external validation through comparison with other data sources. The 12.1× difference between tool-augmented CLI sessions (106.4 msgs) and web chat sessions (8.8 msgs) demonstrates that interaction modality fundamentally shapes collaboration patterns, not just AI capability.

\subsection{Advanced Multi-Analysis Summary}

\begin{figure}[h]
\centering
\includegraphics[width=\linewidth]{figures/advanced_analysis.png}
\caption{Comprehensive Analysis Dashboard: Multi-panel visualization summarizing all key findings including temporal evolution, complexity clusters, model selection patterns, and maturity relationships in a unified analytical framework.}
\label{fig:advanced-summary}
\end{figure}

Figure~\ref{fig:advanced-summary} presents a comprehensive dashboard that integrates all our analytical dimensions. This unified visualization allows researchers to see the interconnections between temporal patterns, complexity archetypes, model optimization strategies, and project evolution trajectories.

\subsection{Heat Map Analysis}

To provide additional visual insights, we generated correlation heat maps for key metrics:

\begin{table}[h]
\centering
\caption{Correlation Heat Map: Key Metrics Relationships}
\label{tab:correlation-heatmap}
\begin{tabular}{l|cccccc}
\toprule
\textbf{Metric} & \textbf{Sessions} & \textbf{Messages} & \textbf{Tool Intensity} & \textbf{Model Tier} & \textbf{Efficiency} & \textbf{Maturity} \\
\midrule
Sessions & 1.00 & 0.85 & 0.72 & 0.68 & 0.45 & 0.91 \\
Messages & 0.85 & 1.00 & 0.66 & 0.73 & 0.38 & 0.83 \\
Tool Intensity & 0.72 & 0.66 & 1.00 & 0.41 & 0.89 & 0.69 \\
Model Tier & 0.68 & 0.73 & 0.41 & 1.00 & 0.32 & 0.65 \\
Efficiency & 0.45 & 0.38 & 0.89 & 0.32 & 1.00 & 0.52 \\
Maturity & 0.91 & 0.83 & 0.69 & 0.65 & 0.52 & 1.00 \\
\bottomrule
\end{tabular}
\end{table}

The correlation analysis reveals strong relationships between project maturity and session metrics (0.91 for sessions, 0.83 for messages), and a particularly strong correlation between tool intensity and efficiency (0.89), supporting our weighted delegation approach.

\subsection{Advanced Heat Map Analysis}

\begin{figure}[h]
\centering
\includegraphics[width=\linewidth]{figures/comprehensive_heatmap_dashboard.png}
\caption{Comprehensive Heat Map Dashboard: Multi-panel visualization showing temporal patterns, complexity archetypes, model selection, tool usage, project maturity, and metric correlations. This unified dashboard reveals the interconnected patterns of cyborg cognition across all analytical dimensions.}
\label{fig:comprehensive-dashboard}
\end{figure}

Figure~\ref{fig:comprehensive-dashboard} presents our most advanced visualization---a comprehensive heat map dashboard that integrates all analytical dimensions into a unified analytical framework. Each panel reveals different aspects of cyborg cognition:

\textbf{Panel 1 - Temporal Evolution}: Shows the efficiency paradox and intensity patterns across weeks, with the dramatic efficiency drop in weeks 5-6 indicating increased complexity handling.

\textbf{Panel 2 - Complexity Archetypes}: Reveals how the four identified clusters differ across key characteristics, with Deep-Dive and Intensive clusters showing distinct patterns.

\textbf{Panel 3 - Model Selection}: Demonstrates how different AI models are optimized for different project characteristics, revealing sophisticated user strategies.

\textbf{Panel 4 - Tool Usage Patterns}: Shows the relationship between usage frequency and cognitive delegation, with exploration and execution tools dominating.

\textbf{Panel 5 - Project Maturity}: Reveals the distribution of projects across maturity stages and their relative efficiency levels.

\textbf{Panel 6 - Metrics Correlation}: Shows the interrelationships between key metrics, with particularly strong correlations between tool intensity and efficiency (0.89), and between maturity and session metrics (0.91 for sessions, 0.83 for messages).

\subsection{Individual Heat Map Analysis}

\begin{figure}[h]
\centering
\includegraphics[width=0.8\linewidth]{figures/heatmap_temporal.png}
\caption{Temporal Heat Map: Detailed view of efficiency, session intensity, and message intensity patterns across the study period, revealing the three-phase adoption pattern with the efficiency paradox clearly visible.}
\label{fig:temporal-heatmap}
\end{figure}

\begin{figure}[h]
\centering
\includegraphics[width=0.8\linewidth]{figures/heatmap_complexity.png}
\caption{Complexity Archetypes Heat Map: Multi-dimensional view of the four project clusters across key characteristics, showing distinct patterns that support our machine learning-based classification.}
\label{fig:complexity-heatmap}
\end{figure}

\begin{figure}[h]
\centering
\includegraphics[width=0.8\linewidth]{figures/heatmap_tool_usage.png}
\caption{Tool Usage Pattern Heat Map: Relationship between usage frequency and cognitive delegation across tool categories, revealing the weighted delegation approach that accounts for cognitive load differences.}
\label{fig:tool-heatmap}
\end{figure}

\begin{figure}[h]
\centering
\includegraphics[width=0.8\linewidth]{figures/heatmap_maturity.png}
\caption{Project Maturity Heat Map: Distribution of projects across maturity stages and their relative efficiency levels, showing the progression from experimental to optimized collaboration.}
\label{fig:maturity-heatmap}
\end{figure}

These individual heat maps provide detailed views of specific analytical dimensions, allowing researchers to examine the granular patterns that support our theoretical framework.

Our enhanced visualizations reveal several critical patterns:

\textbf{1. Non-Linear Evolution}: The temporal analysis shows that cyborg cognition development is not linear but follows predictable phases with temporary efficiency decreases.

\textbf{2. Clustered Complexity}: Projects naturally fall into distinct archetypes, suggesting that one-size-fits-all AI tools may be suboptimal.

\textbf{3. Strategic Model Selection}: Developers exhibit sophisticated optimization behavior, matching AI capabilities to project characteristics and cost constraints.

\textbf{4. Maturity-Based Optimization}: Efficiency gains are most pronounced in mature collaborations, suggesting value in long-term AI integration.

\textbf{5. Context Fluidity}: The ability to rapidly adapt across project contexts is a hallmark of advanced cyborg cognition.

These visualizations provide the empirical foundation for our theoretical contributions and offer practical insights for tool designers and researchers studying human-AI collaboration.

\bibliographystyle{ACM-Reference-Format}
\bibliography{references}

\appendix
% Appendix

\section{Dataset Availability}

\subsection{Aggregated Data}

Aggregated metrics, including session summaries, tool usage distributions, and temporal patterns, are available at:

\begin{quote}
\url{https://github.com/anderson-ufrj/cyborg}
\end{quote}

\subsection{Raw Interaction Logs}

Raw JSONL interaction logs contain potentially sensitive project information (file paths, code snippets, proprietary logic). These are available upon request with appropriate data use agreements. Contact the first author.

\section{Tool Category Definitions}

\subsection{Exploration Tools}
Tools that gather information without modifying state:
\begin{itemize}
    \item \textbf{Read}: Read file contents
    \item \textbf{Grep}: Search for patterns in files
    \item \textbf{Glob}: Find files matching patterns
    \item \textbf{WebSearch}: Search the web
    \item \textbf{WebFetch}: Retrieve web page content
    \item \textbf{LSP}: Language server queries (definitions, references)
\end{itemize}

\subsection{Modification Tools}
Tools that change file contents:
\begin{itemize}
    \item \textbf{Write}: Create or overwrite files
    \item \textbf{Edit}: Apply targeted edits to files
    \item \textbf{MultiEdit}: Apply multiple edits in one operation
    \item \textbf{NotebookEdit}: Edit Jupyter notebook cells
\end{itemize}

\subsection{Execution Tools}
Tools that run commands or manage processes:
\begin{itemize}
    \item \textbf{Bash}: Execute shell commands
    \item \textbf{Task}: Launch sub-agents for complex tasks
    \item \textbf{TaskOutput}: Retrieve output from background tasks
    \item \textbf{KillShell}: Terminate running processes
\end{itemize}

\subsection{Planning Tools}
Tools that manage task state and workflow:
\begin{itemize}
    \item \textbf{TodoWrite}: Create and update task lists
    \item \textbf{EnterPlanMode}: Begin structured planning
    \item \textbf{ExitPlanMode}: Complete planning phase
\end{itemize}

\subsection{Interaction Tools}
Tools that explicitly request human input:
\begin{itemize}
    \item \textbf{AskUserQuestion}: Pose questions to the developer
\end{itemize}

\section{Delegation Score Calculation}

The delegation score $D$ is computed as a weighted average of tool category usage:

\begin{equation}
D = \frac{\sum_{c \in C} p_c \cdot w_c}{\sum_{c \in C} p_c}
\end{equation}

Where:
\begin{itemize}
    \item $C$ = set of tool categories
    \item $p_c$ = percentage of tool uses in category $c$
    \item $w_c$ = delegation weight for category $c$
\end{itemize}

Delegation weights:
\begin{itemize}
    \item Exploration: $w = 1.0$ (high delegation---AI gathers information)
    \item Planning: $w = 1.0$ (high delegation---AI manages tasks)
    \item Modification: $w = 0.5$ (medium---shared authorship)
    \item Execution: $w = 0.5$ (variable---depends on command autonomy)
    \item Interaction: $w = 0.0$ (low---human deciding)
\end{itemize}

For our dataset:
\begin{align}
D &= \frac{33.5 \times 1.0 + 8.7 \times 1.0 + 21.5 \times 0.5 + 35.8 \times 0.5 + 0.5 \times 0.0}{100} \\
&= \frac{33.5 + 8.7 + 10.75 + 17.9 + 0}{100} \\
&= \frac{70.85}{100} \\
&= 0.71
\end{align}

\section{Model Tier Definitions}

\begin{table}[h]
\caption{Claude Model Tiers}
\begin{tabular}{llll}
\toprule
\textbf{Model} & \textbf{Tier} & \textbf{Characteristics} & \textbf{Typical Use} \\
\midrule
Opus 4.5 & High & Complex reasoning, large context & Architecture, debugging \\
Sonnet 4.5 & Medium & Balanced capability and speed & General development \\
Haiku 4.5 & Low & Fast, efficient & Quick queries, routine \\
\bottomrule
\end{tabular}
\end{table}

\section{Session JSONL Format}

Each session is stored as a JSONL file with events:

\begin{lstlisting}[caption={Sample JSONL event}]
{
  "type": "assistant",
  "message": {
    "role": "assistant",
    "model": "claude-opus-4-5-20251101",
    "content": [...],
    "usage": {
      "input_tokens": 1234,
      "output_tokens": 567
    }
  },
  "timestamp": "2025-12-30T10:15:00.000Z",
  "sessionId": "abc-123-def",
  "cwd": "/home/user/project"
}
\end{lstlisting}

Tool invocations appear within message content:

\begin{lstlisting}[caption={Tool use structure}]
{
  "type": "tool_use",
  "id": "toolu_01ABC...",
  "name": "Bash",
  "input": {
    "command": "npm test"
  }
}
\end{lstlisting}


\end{document}
